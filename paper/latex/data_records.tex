\section{Data Records}
\label{sec:records}

The Science Data Lake is hosted on HuggingFace Datasets
(\textbf{[URL to be inserted upon publication]}), which provides a persistent
DataCite DOI for citation. Remote users can query the Parquet files directly
through DuckDB's \texttt{hf://} protocol without downloading the full dataset.

The dataset comprises approximately 960\,GB of compressed Apache Parquet files
organized into 22~schema directories, each containing one or more Parquet files
corresponding to the tables of that schema. The lightweight DuckDB database file
({\raise.17ex\hbox{$\scriptstyle\sim$}}500\,KB) defines 151~SQL views that
reference these Parquet files and can be regenerated from source using the
provided pipeline scripts.

The principal schemas and their contents are:

\begin{itemize}[nosep]
  \item \texttt{s2ag}: papers (231M), abstracts, citations, authors, publication
    venues from Semantic Scholar.
  \item \texttt{openalex}: works (479M), authors, institutions, sources, topics,
    concepts, publishers, funders from OpenAlex.
  \item \texttt{sciscinet}: paper metrics (250M), disruption indices, atypicality
    scores, team size indicators from SciSciNet.
  \item \texttt{pwc}: papers (513K), code links, tasks, datasets, methods from
    Papers with Code.
  \item \texttt{retwatch}: retracted papers (69K) with retraction reasons and
    dates from Retraction Watch.
  \item \texttt{ros}: patent--paper citation pairs (47.8M) from Reliance on Science.
  \item \texttt{p2p}: preprint-to-published DOI mappings (146K) from bioRxiv/medRxiv.
  \item \texttt{xref}: cross-source linkage tables---\texttt{unified\_papers}
    (293M, 29~columns), \texttt{doi\_map}, and \texttt{topic\_ontology\_map}
    (16,150 mappings).
  \item 13~ontology schemas (e.g., \texttt{mesh}, \texttt{go}, \texttt{cso}),
    each containing \texttt{*\_terms}, \texttt{*\_hierarchy}, and optionally
    \texttt{*\_xrefs} tables.
\end{itemize}

Each source retains its original license: CC0 (OpenAlex), ODC-BY (S2AG),
CC~BY 4.0 (SciSciNet), CC~BY-SA 4.0 (Papers with Code), CC~BY-NC 4.0
(Reliance on Science), and open/public-domain for the remaining sources.
Users should comply with the most restrictive license applicable to the
sources they query.
