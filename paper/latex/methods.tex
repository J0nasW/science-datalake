\section{Methods}
\label{sec:methods}

\subsection{Data Sources}
\label{sec:sources}

The Science Data Lake integrates eight open scholarly data sources, each contributing
distinct metadata facets (Table~\ref{tab:sources}).

\begin{table}[t]
\centering
\caption{Overview of integrated data sources. Record counts reflect the snapshot
versions used in this release.}
\label{tab:sources}
\small
\begin{tabularx}{\textwidth}{@{}l r l X@{}}
\toprule
\textbf{Source} & \textbf{Records} & \textbf{License} & \textbf{Key content} \\
\midrule
Semantic Scholar (S2AG)     & 231M  & ODC-BY   & Citations, influential citations, open access \\
OpenAlex                    & 479M  & CC0      & FWCI, topics, types, languages, funding awards \\
SciSciNet                   & 250M  & CC BY    & Disruption, atypicality, team size \\
Papers with Code            & 513K  & CC BY-SA & Code repositories, tasks, datasets \\
Retraction Watch            & 69K   & Open     & Retraction reasons and dates \\
Reliance on Science (RoS)   & 47.8M & CC BY-NC & Patent--paper citation pairs \\
Preprint-to-Published (P2P) & 146K  & Open     & bioRxiv/medRxiv DOI to published DOI \\
Crossref                    & ---   & Open     & DOI metadata, reference lists \\
\bottomrule
\end{tabularx}
\end{table}

\textbf{Semantic Scholar Academic Graph (S2AG)}~\cite{kinney2023s2ag} provides
bibliometric metadata for approximately 231~million papers, including citation counts,
influential citation counts (based on citation context analysis), and open-access
status flags.

\textbf{OpenAlex}~\cite{priem2022openalex} is an open catalogue of 479~million
scholarly works with field-weighted citation impact (FWCI), a four-level topic taxonomy
(domain, field, subfield, topic with 4,516~leaf topics), document types, and language
annotations.

\textbf{SciSciNet}~\cite{lin2023sciscinet} augments OpenAlex records with
pre-computed science-of-science metrics including the disruption index
CD\textsubscript{5}~\cite{funk2017disruption,wu2019disruption}, journal atypicality
$z$-scores~\cite{uzzi2013atypicality}, team size indicators, and patent citation
counts for 250~million papers.

\textbf{Papers with Code}~\cite{paperswithcode2024} links 513~thousand machine-learning
papers to their associated code repositories, benchmark tasks, and datasets.

\textbf{Retraction Watch}~\cite{retractionwatch2024} catalogues approximately 69~thousand
retracted or corrected publications with structured retraction reasons and dates.

\textbf{Reliance on Science (RoS)}~\cite{marx2020ros} provides 47.8~million patent-to-paper citation pairs from global patent
offices, with confidence scores and citation location metadata.

\textbf{Preprint-to-Published (P2P)} provides approximately 146~thousand mappings from
bioRxiv and medRxiv preprint DOIs to their corresponding published-version DOIs.

\textbf{Crossref} contributes DOI metadata and reference lists used for DOI
validation and supplementary linkage.

Figure~\ref{fig:temporal} shows the temporal coverage of each source, revealing
structural differences: OpenAlex extends back several centuries, S2AG is concentrated
in recent decades with a computer-science emphasis, and SciSciNet metrics end around
2022.

\begin{figure}[t]
\centering
\includegraphics[width=\textwidth]{figures/fig3_temporal_ridgeline.pdf}
\caption{Temporal coverage by source (symlog scale). Publication-year distributions
for DOI-linked records in the unified index (see Table~\ref{tab:sources} for full
source sizes). The symlog y-axis is linear near zero and logarithmic above, revealing
both historical depth and recent growth. OpenAlex, S2AG, and SciSciNet extend back to
1900 with tens of thousands of papers per year, while all three show clear acceleration
after $\sim$1960. SciSciNet exhibits a sharp cutoff after $\sim$2022 when its metrics
computation ends. The specialized sources (PWC, Retraction Watch) cover narrow temporal
windows concentrated in the last two decades.}
\label{fig:temporal}
\end{figure}

% ============================================================
\subsection{Architecture}
\label{sec:architecture}

The Science Data Lake is built on a \emph{views-over-Parquet} architecture using
DuckDB~\cite{raasveldt2019duckdb} (Figure~\ref{fig:architecture}). Each data source
is first converted from its native format (JSON Lines, CSV, N-Triples) into columnar
Apache Parquet files, totaling approximately 960\,GB on disk. A lightweight DuckDB
database ({\raise.17ex\hbox{$\scriptstyle\sim$}}270\,KB) defines 151~SQL views
organized into 22~schemas that reference these Parquet files without copying data.

The schema design follows two principles. \textbf{Source-level preservation}: each
data source retains its native schema within a dedicated namespace (e.g.,
\texttt{s2ag.papers}, \texttt{openalex.works}, \texttt{sciscinet.paper\_metrics}),
enabling direct inspection of how different databases represent the same paper.
\textbf{Cross-referencing via the \texttt{xref} schema}: three materialized views
bridge across sources---\texttt{doi\_map} (DOI normalization), \texttt{unified\_papers}
(293M-row join table), and \texttt{topic\_ontology\_map} (ontology alignment).

We chose DuckDB over traditional server-based systems such as PostgreSQL for
three reasons: its embeddable, serverless design requires no installation or
administration; its columnar engine is optimized for the analytical (OLAP) query
patterns typical of bibliometric research; and---crucially---it reads Parquet
files directly through SQL views without importing data, keeping the database
file under 300\,KB while exposing nearly 1\,TB of data.

The system supports dual-mode access: local deployment on a high-capacity drive
for full-speed analytical queries, or remote access through HuggingFace-hosted
Parquet files for users without local storage.

A reproducible pipeline orchestrated by a master CLI script (\texttt{datalake\_cli.py})
automates the full workflow: downloading source snapshots, converting to Parquet,
creating DuckDB views, materializing cross-reference tables, and building the ontology
linkage.

% Architecture diagram — included from methods.tex via \input
\begin{figure}[t]
\centering
\resizebox{\textwidth}{!}{%
\begin{tikzpicture}[
  % Node styles
  source/.style={
    rectangle, rounded corners=2pt, minimum width=3.2cm, minimum height=0.65cm,
    draw=sourceblue!70, fill=sourceblue!8, font=\small\sffamily,
    text=black, line width=0.5pt, align=center
  },
  countlabel/.style={font=\scriptsize\sffamily\color{gray!70!black}, anchor=west},
  parquetnode/.style={
    rectangle, rounded corners=3pt, minimum width=2.4cm, minimum height=2.0cm,
    draw=parquetgray!70, fill=parquetgray!6, font=\small\sffamily\bfseries,
    text=parquetgray!90!black, line width=0.6pt, align=center
  },
  schemanode/.style={
    rectangle, rounded corners=2pt, minimum width=3.0cm, minimum height=0.55cm,
    draw=duckdbyellow!70!black, fill=duckdbyellow!12,
    font=\small\ttfamily, text=black, line width=0.5pt
  },
  xrefnode/.style={
    rectangle, rounded corners=2pt, minimum width=2.8cm, minimum height=0.55cm,
    draw=xreforange!70!black, fill=xreforange!12,
    font=\small\ttfamily, text=xreforange!85!black, line width=0.5pt
  },
  ontonode/.style={
    rectangle, rounded corners=2pt, minimum width=2.8cm, minimum height=0.55cm,
    draw=ontopurple!60, fill=ontopurple!8, font=\small\sffamily,
    text=ontopurple!85!black, line width=0.5pt
  },
  groupframe/.style={
    rectangle, rounded corners=5pt, draw=#1!40, fill=#1!3,
    line width=0.7pt, inner sep=8pt
  },
  grouplabel/.style={font=\small\sffamily\bfseries, text=#1!75!black},
  myarrow/.style={-{Stealth[length=4pt,width=3pt]}, line width=0.6pt, color=#1!60!black},
  thickarrow/.style={-{Stealth[length=5pt,width=4pt]}, line width=0.9pt, color=#1!60!black},
  note/.style={font=\scriptsize\sffamily, text=gray!65!black},
]

% ============================================================
% COLUMN 1: Data Sources (x=0, spaced at 1.0 intervals)
% ============================================================
\node[source] (s2ag)    at (0, 0)    {Semantic Scholar};
\node[source] (oalex)   at (0,-1.0)  {OpenAlex};
\node[source] (ssn)     at (0,-2.0)  {SciSciNet};
\node[source] (pwc)     at (0,-3.0)  {Papers with Code};
\node[source] (retw)    at (0,-4.0)  {Retraction Watch};
\node[source] (ros)     at (0,-5.0)  {Reliance on Science};
\node[source] (p2p)     at (0,-6.0)  {Preprint-to-Published};
\node[source] (xrefsrc) at (0,-7.0)  {Crossref};

% Record counts (anchored left, positioned outside the source group frame)
\node[countlabel] at (2.0, 0.2)    {231M};
\node[countlabel] at (2.0,-0.7)  {479M};
\node[countlabel] at (2.0,-1.7)  {250M};
\node[countlabel] at (2.0,-2.7)  {513K};
\node[countlabel] at (2.0,-3.7)  {69K};
\node[countlabel] at (2.0,-4.7)  {47.8M};
\node[countlabel] at (2.0,-5.7)  {146K};

% Source group label
\node[grouplabel=sourceblue, anchor=south] at (0, 0.7) {Data Sources};

% ============================================================
% COLUMN 2: Conversion bus + Parquet (x=3.8..6.0)
% ============================================================

% Vertical bus line (all sources feed into this)
\coordinate (bustop)    at (3.5, 0);
\coordinate (busbottom) at (3.5, -7.0);
\draw[parquetgray!50, line width=1.5pt] (bustop) -- (busbottom);

% Short arrows from each source to the bus
\foreach \src in {s2ag,oalex,ssn,pwc,retw,ros,p2p,xrefsrc} {
  \draw[myarrow=parquetgray] (\src.east) -- (3.5, 0 |- \src.east);
}

% Parquet node (centered on the bus vertically)
\node[parquetnode] (pq) at (5, -3.5)
  {Parquet\\[2pt]\normalfont\scriptsize{\raise.17ex\hbox{$\scriptstyle\sim$}}960\,GB};

% Single thick arrow from bus to parquet
\draw[thickarrow=parquetgray] (3.5, -3.5) -- (pq.west);

% ============================================================
% COLUMN 3 upper: DuckDB Source Schemas (x=10.0)
% ============================================================

% DuckDB title + subtitle
\node[grouplabel=duckdbyellow, anchor=south] at (10.0, 1.0)
  {DuckDB (22 schemas, 151 views)};
\node[note, anchor=north] at (10.0, 0.9)
  {view-only DB {\raise.17ex\hbox{$\scriptstyle\sim$}}500\,KB};

% Source schema nodes (starting at y=0, spaced at 0.80)
\node[schemanode] (ds1) at (10.0, 0.0)  {s2ag.*};
\node[schemanode] (ds2) at (10.0,-0.80) {openalex.*};
\node[schemanode] (ds3) at (10.0,-1.60) {sciscinet.*};
\node[schemanode] (ds4) at (10.0,-2.40) {pwc.*, retwatch.*};
\node[note]       at (10.0,-3.10)       {\textit{+ ros, p2p, main}};

% ============================================================
% COLUMN 3 lower: xref Schema (x=10.0, starting at y=-4.5)
% ============================================================

% xref group label
\node[grouplabel=xreforange, anchor=south] at (10.0, -3.85)
  {\texttt{xref} schema};

\node[xrefnode, minimum width=3.6cm, minimum height=0.65cm] (unified) at (10.0, -4.55)
  {unified\_papers};
\node[note, below=1pt of unified] {293M rows, 29 columns};

\node[xrefnode, minimum width=2.4cm] (doimap) at (10, -5.8) {doi\_map};
\node[xrefnode, minimum width=3.0cm] (tomap)  at (10, -6.5) {topic\_ontology\_map};

% ============================================================
% COLUMN 4: Ontology Layer (x=15.5)
% ============================================================

% Ontology group label
\node[grouplabel=ontopurple, anchor=south] at (15.5, -3.85)
  {13 Ontology Schemas};

\node[ontonode] (mesh)  at (15.5, -4.55) {MeSH (721K)};
\node[ontonode] (chebi) at (15.5, -5.30) {ChEBI (205K)};
\node[ontonode] (go)    at (15.5, -6.05) {GO (48K)};
\node[ontonode] (cso)   at (15.5, -6.80) {CSO (15K)};
\node[note]     at (15.5, -7.30) {\textit{+ 9 more ontologies}};
\node[note]     at (15.5, -7.70) {1.3M terms total};

% ============================================================
% Background group frames
% ============================================================
\begin{scope}[on background layer]
  % Source frame
  \node[groupframe=sourceblue,
    fit=(s2ag)(xrefsrc), inner sep=12pt] {};

  % DuckDB frame (encompasses source schemas + xref)
  \node[groupframe=duckdbyellow,
    fit=(ds1)(ds4)(unified)(doimap)(tomap),
    inner sep=14pt, inner ysep=18pt] (duckdb) {};

  % Ontology frame
  \node[groupframe=ontopurple,
    fit=(mesh)(cso), inner xsep=14pt, inner ysep=10pt] (ontology) {};
\end{scope}

% ============================================================
% Connecting arrows
% ============================================================

% Parquet → DuckDB schemas
\draw[thickarrow=duckdbyellow] (pq.east) -- ++(0.4,0)
  node[above, note, pos=0.5] {} |- (duckdb.west);

% \draw[thickarrow=duckdbyellow] (pq.east) -- (duckdb.west);

% Source schemas → unified_papers (dashed)
\draw[myarrow=xreforange, densely dashed]
  ([xshift=6pt]ds4.south) -- ++(0,-0.55) -| ([xshift=-4pt]unified.north);

% topic_ontology_map → ontologies
\draw[thickarrow=ontopurple] (tomap.east) -- ++(1.2,0)
  node[above, note, pos=0.4] {BGE-large}
  node[below, note, pos=0.4] {embeddings} |- (ontology.west);


\end{tikzpicture}%
}
\caption{Architecture of the Science Data Lake.
Eight open scholarly data sources (left) are converted to Apache Parquet format
({\raise.17ex\hbox{$\scriptstyle\sim$}}960\,GB) and exposed as SQL views through
a lightweight DuckDB database (center). Each source retains its native schema for
source-level fidelity. The cross-referencing \texttt{xref} schema (orange) links
records via DOI normalization (\texttt{unified\_papers}, 293M rows) and connects
OpenAlex topics to 13~scientific ontologies (right) through BGE-large
embedding-based alignment.}
\label{fig:architecture}
\end{figure}


% ============================================================
\subsection{DOI Normalization and Record Linkage}
\label{sec:doi}

Digital Object Identifiers (DOIs) serve as the primary key for cross-source record
linkage, but sources store them in incompatible formats (Table~\ref{tab:doi}).

\begin{table}[t]
\centering
\caption{DOI format differences across data sources and the normalization applied.}
\label{tab:doi}
\small
\begin{tabular}{@{}l l l@{}}
\toprule
\textbf{Source} & \textbf{Raw DOI format} & \textbf{Normalization} \\
\midrule
S2AG             & lowercase, no prefix (\texttt{10.1038/\ldots}) & Canonical (none) \\
OpenAlex         & lowercase, \texttt{https://doi.org/} prefix    & Strip prefix \\
SciSciNet        & lowercase, \texttt{https://doi.org/} prefix    & Strip prefix \\
Papers with Code & lowercase, no prefix                           & None \\
Retraction Watch & lowercase, no prefix                           & None \\
Crossref         & mixed case                                     & Lowercase \\
\bottomrule
\end{tabular}
\end{table}

All DOIs are normalized to a canonical lowercase, prefix-free format. The
\texttt{xref.doi\_map} view implements this normalization as a union of
source-specific sub-queries, each applying the appropriate transformation.

The resulting \texttt{xref.unified\_papers} table contains 293,123,121 unique DOIs
with 29~columns drawn from all sources, including six Boolean coverage flags
indicating which sources contain each paper. Table~\ref{tab:overlap} summarizes the
pairwise coverage.

\begin{table}[t]
\centering
\caption{Cross-source coverage of the 293M unified papers. Each cell shows the
percentage of papers present in the column source that are also present in the
row source.}
\label{tab:overlap}
\small
\begin{tabular}{@{}l r r r r r r@{}}
\toprule
 & \textbf{OpenAlex} & \textbf{SciSciNet} & \textbf{S2AG} & \textbf{PWC} & \textbf{RetWatch} & \textbf{RoS} \\
\midrule
Coverage (\%) & 99.67 & 54.08 & 45.52 & 0.048 & 0.020 & 0.19 \\
\bottomrule
\end{tabular}
\end{table}

OpenAlex provides the broadest coverage at 99.67\% of all DOIs, consistent with
its role as a comprehensive open index. SciSciNet and S2AG cover approximately
half the DOI space, reflecting their focus on papers with sufficient citation
data for metric computation. The specialized sources (Papers with Code, Retraction
Watch, Reliance on Science) contribute smaller but unique record sets that cannot
be obtained from the three large databases.

Figure~\ref{fig:upset} shows the UpSet plot of the six-source overlap, revealing
34~observed source combinations. The dominant combination is OpenAlex-only
(45.0\%), followed by the three-way overlap of OpenAlex, SciSciNet, and S2AG
(38.2\%).

\begin{figure}[t]
\centering
\includegraphics[width=\textwidth]{figures/fig2_upset_source_overlap.pdf}
\caption{UpSet plot showing the intersection structure across six data sources.
Bars represent the number of papers in each source combination. Of 34~observed
combinations, the three-way overlap of OpenAlex, SciSciNet, and S2AG accounts
for the largest multi-source intersection.}
\label{fig:upset}
\end{figure}

% ============================================================
\subsection{Embedding-Based Ontology Alignment}
\label{sec:ontology}

OpenAlex assigns papers to a flat topic taxonomy of 4,516~topics organized into
four hierarchical levels (252~subfields, 26~fields, 4~domains), but these topics
lack mappings to formal scientific ontologies that encode domain-specific
knowledge. To bridge this gap, we developed an embedding-based alignment method
that maps OpenAlex topics to 13~scientific ontologies comprising 1.3~million
terms in total.

The 13~ontologies span diverse scientific domains: Medical Subject Headings
(MeSH; 721K terms)~\cite{lipscomb2000mesh}, Chemical Entities of Biological
Interest (ChEBI; 205K)~\cite{hastings2016chebi}, NCI Thesaurus (NCIT;
204K)~\cite{sioutos2007ncit}, Gene Ontology (GO;
48K)~\cite{go2021geneontology}, AGROVOC
(42K)~\cite{caracciolo2013agrovoc}, Computer Science Ontology (CSO;
15K)~\cite{salatino2020cso}, Disease Ontology
(DOID)~\cite{schriml2022doid}, Human Phenotype Ontology
(HPO)~\cite{kohler2021hpo}, EDAM bioinformatics
ontology~\cite{ison2013edam}, UNESCO
Thesaurus\footnote{\url{https://vocabularies.unesco.org/browser/thesaurus/en/}},
Standard Thesaurus for Economics
(STW)\footnote{\url{https://zbw.eu/stw/}}, Physics Subject Headings
(PhySH)\footnote{\url{https://physh.org/}}, and the Mathematics Subject
Classification (MSC2020)\footnote{\url{https://msc2020.org/}}. Each ontology was converted
from its native format (OBO, SKOS/RDF, N-Triples, CSV) to a uniform Parquet
representation using format-specific parsers, and simultaneously loaded into an
Oxigraph RDF triple store for SPARQL queries.

We employed a hybrid alignment strategy. For the 10~smaller ontologies (291K
terms including synonyms), we computed dense embeddings using the BGE-large-en-v1.5
model~\cite{xiao2024bge} (335M parameters, 1024 dimensions) and performed
nearest-neighbour search via a FAISS index~\cite{johnson2021faiss}. For the three largest ontologies (MeSH, ChEBI, NCIT), which
together account for 1.1M terms and would dominate the embedding space, we used
exact string matching to ensure precision.

Table~\ref{tab:ontology_tiers} summarizes the alignment quality with
representative examples for each tier. At the exact-match tier ($\geq 0.95$),
topics align with near-identical ontology terms---for instance, the OpenAlex
topic ``Machine Learning'' maps to CSO's ``machine learning'' (similarity
0.98). At the high-quality tier ($\geq 0.85$), semantically related but
differently named concepts are captured---e.g., ``Artificial Intelligence in
Medicine'' maps to EDAM's ``Medical informatics'' (0.87). Relaxing to
$\geq 0.65$ yields 16,150~mappings covering 4,509 of 4,516 topics (99.84\%),
including broader associations such as ``Soil Chemistry'' mapping to
AGROVOC's ``soil chemicophysical properties'' (0.68).

\begin{table}[t]
\centering
\caption{Ontology alignment quality tiers. Each tier includes all mappings at or
above the similarity threshold.}
\label{tab:ontology_tiers}
\small
\begin{tabular}{@{}l r r r@{}}
\toprule
\textbf{Quality tier} & \textbf{Similarity} & \textbf{Mappings} & \textbf{Topics covered} \\
\midrule
Exact match   & $\geq 0.95$ &     85 & 71 (1.6\%) \\
High quality  & $\geq 0.85$ &  2,527 & 1,647 (36.5\%) \\
All           & $\geq 0.65$ & 16,150 & 4,509 (99.84\%) \\
\bottomrule
\end{tabular}
\end{table}

To contextualize the embedding approach, we compared it against a string-matching
baseline using Jaro--Winkler similarity at a threshold of 0.90, which produced only
937~matches---a 17-fold reduction relative to the embedding method. The embedding
approach captures semantic similarity that string matching cannot: for example,
the OpenAlex topic ``Artificial Intelligence in Medicine'' maps to EDAM's
``Medical informatics'' (cosine similarity 0.87) and NCIT's ``Biomedical
Informatics'' (0.85), neither of which would be found by string comparison.

Figure~\ref{fig:umap} visualizes the joint embedding space using
UMAP~\cite{mcinnes2018umap}, showing how OpenAlex topics cluster by domain and
align with terms from domain-specific ontologies. Figure~\ref{fig:heatmap}
displays the ontology-by-domain reach heatmap, confirming that different
ontologies specialize in different scientific areas: MeSH dominates health
sciences, CSO covers computer science, GO spans molecular biology, and AGROVOC
bridges agricultural and environmental sciences.

\begin{figure}[t]
\centering
\includegraphics[width=\textwidth]{figures/fig4_umap_embedding_space.pdf}
\caption{UMAP projection of BGE-large embeddings for OpenAlex topics (points)
and matched ontology terms (crosses), colored by OpenAlex domain. Semantic
clusters emerge naturally, with domain-specific ontology terms co-locating with
their corresponding topics.}
\label{fig:umap}
\end{figure}

\begin{figure}[t]
\centering
\includegraphics[width=\textwidth]{figures/fig5_ontology_reach_heatmap.pdf}
\caption{Ontology reach heatmap showing the number of high-quality mappings
($\text{similarity} \geq 0.85$) between each ontology and each OpenAlex domain.
The multi-ontology design ensures coverage across all scientific areas.}
\label{fig:heatmap}
\end{figure}
